\documentclass[12pt]{article}
\usepackage[T1]{fontenc}
\usepackage{graphicx}
\usepackage{float}
\usepackage[polish]{babel}
\usepackage{amsmath}

\setlength{\textheight}{21cm}

\title{{\bf Zadanie nr 3 - Splot, filtracja i korelacja sygnałów}\linebreak
Cyfrowe Przetwarzanie Sygnałów}
\author{Jakub Wąchała, 216914 \and Radosław Grela, 216769}
\date{17.05.2020}

\begin{document}
\clearpage\maketitle
\thispagestyle{empty}
\newpage
\setcounter{page}{1}
\section{Cel zadania}
\label{cel}
Celem zadania jest oswojenie się z zagadnieniami dotyczącymi splotu, filtracji i korelacji sygnałów. Zadanie polega na implementacji wybranych wariantów filtracji, funkcji okien, które są często wykorzystywane w praktyce cyfrowej filtracji sygnałów. 

\section{Wstep teoretyczny}
Program ten jest wzbogaconą o powyższe funkcjonalności wersją programu z zadania 1. i 2. Umożliwia wykonanie operacji splotu, korelację sygnałów dyskretnych, tworzenie filtrów o ustalonej wartości (M - rząd filtru, f0 - odcięcia filtru, fd - częstotliwość próbkowania sygnału) z wykorzystaniem okien. 
\begin{itemize}
\item Splot - jedna z najważniejszych operacji, która wykorzystywana jest podczas filtracji sygnałów dyskretnych. Polega na przetwarzaniu dwóch sygnałów dyskretnych co w konsekwencji daje nam jeden sygnał dyskretny.
\item Korelacja sygnałów - bardzo ważna rzecz w przetwarzaniu sygnałów. Używana gdy porównujemy ze sobą dwa sygnały np. sygnał oryginalny z sygnałem oryginalnym, ale przesuniętym na osi. Tak samo jak operacja splotu - podając dwa sygnały otrzymujemy jeden.
\item Filtracja - jedna z podstawowych operacji w cyfrowym przetwarzaniu sygnałów. W jej procesie widmo sygnału podlega modyfikacji tj. odfiltrowanie składowych sygnału, których częstotliwości znajdują się w paśmie zaporowym, natomiast te, które znajdują się w paśmie przepustowym(pozostała część) nie są zmieniane lub ulegają małemu tłumieniu.
\item Okno - postać odpowiedzi impulsowej filtru SOI??????????????
\end{itemize}
\subsection {Dodatkowo zaimplementowane warianty:}
\begin {enumerate}
\item Wykorzystane okna
\begin {itemize}
 \item okno prostokątne
\item (O2) okno Hanninga
\end {itemize}
\item Wykorzystane filtry
\begin {itemize}
\item filtr dolnoprzepustowy
\item (F1) filtr środkowoprzepsutowy
zaimplementowac operacje filtracji podstawiajac odpowiedz impulsowa filtru do wzoru na 
splot ????
\end {itemize}
\item  Operacja splotu splotu 
\item Korelacja sygnałów dyskretnych 
\end{enumerate}

\section{Eksperymenty i wyniki}
Eksperymenty zostały przez nas podzielone na: operacje splotu, operacje korelacji, operacje filtracji z wykorzystaniem okna. Skorzystamy z funkcji trójkątnej i sinusoidalnej z parametrami:
\begin{itemize}
\item amplituda: 15
\item okres: 5
\item czas początkowy: 0
\item czas trwania: 20
\end{itemize}
\subsection{Operacje splotu}
\begin{figure}[H]
\centering
\includegraphics[scale=0.6]{1sinusKwantStopien5.png}
\caption{Operacja splotu funkcji sinusoidalnej}
\end{figure}

\begin{figure}[H]
\centering
\includegraphics[scale=0.6]{1sinusKwantStopien5.png}
\caption{Operacja splotu funkcji trójkątnej}
\end{figure}

\subsection{Operacje korelacji}
\begin{figure}[H]
\centering
\includegraphics[scale=0.6]{1sinusKwantStopien5.png}
\caption{Operacja korelacji bezpośredniej dla funkcji sinusoidalnej}
\end{figure}

\begin{figure}[H]
\centering
\includegraphics[scale=0.6]{1sinusKwantStopien5.png}
\caption{Operacja korelacji przez splot fun sinus???}
\end{figure}

\begin{figure}[H]
\centering
\includegraphics[scale=0.6]{1sinusKwantStopien5.png}
\caption{Operacja korelacji przez splot fun trójkątn???}
\end{figure}

\subsection{Operacje filtracji z wykorzystaniem okna}
Celem eksperymentu jest przedstawienie wyników procesu filtracji z wykorzystaniem okna. O parametrach:
\begin{itemize}
\item Rząd filtru (M): 51
\item Częstotliwość odcięcia filtru (f0): 15
\item Częstotliwość próbkowania sygnału(fd): 200
\end{itemize}

\subsubsection{Rezultat}
\begin{figure}[H]
\centering
\includegraphics[scale=0.6]{8trojkatInterp1rzedu20.png}
\caption{Filtr dolnoprzepustowy z oknem Hanninga}
\end{figure}

\begin{figure}[H]
\centering
\includegraphics[scale=0.6]{8trojkatInterp1rzedu20.png}
\caption{Filtr dolnoprzepustowy z oknem Hanninga dla funkcji sinusoidalnej}
\end{figure}

\begin{figure}[H]
\centering
\includegraphics[scale=0.6]{8trojkatInterp1rzedu20.png}
\caption{Filtr dolnoprzepustowy z oknem Hanninga dla funkcji trójkątnej}
\end{figure}

\begin{figure}[H]
\centering
\includegraphics[scale=0.6]{8trojkatInterp1rzedu20.png}
\caption{Filtr średnioprzepustowy z oknem Hanninga dla funkcji trójkątnej}
\end{figure}

\section{Wnioski}
\begin {itemize}
\item Prezentowane wyniki są dowodem na poprawne wykonanie zadania tj. poprawność konwersji A/C (kwantyzacji równomiernej z zaokrągleniem oraz próbkowania), C/A (interpolacji pierwszego rzędu, ekstrapolacji zerowego rzędu, rekonstrukcji w oparciu o funkcję sinus) wraz z ich miarami. Widzimy, że im większa częstotliwość (większa ilość próbek na sekundę) tym wykres zostaje zrekonstruowany w sposób bardziej dokładny.
\item W przypadku ekstrapolacji widać widoczne zniekształcenie w stosunku do sygnału oryginalnego ponieważ powstaje ona przy użyciu funkcji rect (prostokąt), przez co przybiera ona postać schodkową. Lepiej wygląda to w wypadku interpolacji.
\item Rekonstrukcja w opraciu o funkcję sinus jest dokładniejsza od ekstrapolacji zerowego rzędu (przyjmując, że porównujemy te same częstotliwości)
\item Im większy stopień kwantyzacji tym błąd oraz maksymalna różnica jest mniejsza
\end {itemize}


\begin{thebibliography}{0}
\bibitem{bib1}
\label{zad1}
\textit{Zadanie 2 - Próbkowanie i kwantyzacja} Cyfrowe Przetwarzanie Sygnału WIKAMP FTIMS\newline
\end{thebibliography}

\end{document}